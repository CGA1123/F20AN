\chapter{DNS Rebinding Countermeasures}

There have been attempts made to eradicate DNS rebinding, but only
a few methods have been proven effective at mitigating the effects
or stopping it entirely.

\section{Extended Same-Origin Policy}

With early iterations of the Same-Origin policy, many variations of
DNS rebinding attacks would exploit the policy. To counter this, web
browsers implemented countermeasures (as described in this section) to
protect resources. The Same-Origin policy relies on information
obtained from the DNS, which may not be under control of the owners
of a web server. Therefore a light-weight extension to the policy,
called the Extended Same-Origin policy, was made to also take into
account information provided by a web server to avoid exploitation by
DNS rebinding~\cite{johns2013eradicating}.

\section{DNS Pinning}

DNS Pinning is a technique web browsers implement, which locks an IP
address to the value provided by the initial DNS response. It has
however has been deprecated due to it unintentionally blocking a
few legitimate uses of Dynamic DNS. An example is load balancing,
which is a service provided by DNS. Load balancing is vital for
major web servers and DNS pinning interferes with this. As well as
this, DNS pinning does not protect against \textbf{sophisticated}
DNS rebinding attacks.

\section{DNS Filtering}

Another technique used to prevent DNS rebinding attacks includes
filtering out private IP addresses from DNS responses, which can be
done in a number of ways.

\subsection{Filtering through external DNS servers}

OpenDNS is a service that extends the DNS by adding several security
features, in addition to regular DNS services. One of these features
includes optional content filtering and the ability to filter out
IP addresses from DNS responses. It supports a protocol which
authenticates traffic that moves between a host computer and the
OpenDNS nameservers, called the DNSCrypt protocol.

\subsection{Local nameserver configuration}

System admins can configure their organisation's local nameservers
such that external names cannot be resolved/mapped to the organisation's
internal IP addresses. This is a useful technique to prevent against
DNS rebinding, however a potential attacker could map the internal
IP address ranges that the organisation is currently using.

\subsection{DNS filtering in a firewall}

dnswall?

\section{Router Configuration}

DNS Rebinding attacks have been stopped more successfully through
more secure router configurations. Services, such as the HTTP server
on a router, are bound to all network interfaces and can be accessible
by all IP addresses in a range that it has. The routers drop anything
that comes in through the external port, which doesn't matter because
it accesses external IP addresses from the internal Local Area Network (LAN).
