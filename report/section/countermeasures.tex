\chapter{DNS Rebinding Countermeasures}

There have been attempts made to eradicate DNS rebinding, but only
a few methods have been proven effective at mitigating the effects
or stopping it entirely.

\section{Extended Same-Origin Policy}

https://www.usenix.org/conference/usenixsecurity13/technical-sessions/presentation/johns

\section{DNS Pinning}

Explain what DNS pinning is...

DNS Pinning however has been deprecated. This is because load balancing
provided by DNS is vital for major web servers and DNS pinning
interferes with load balancing.

HF-"However, some sort of verification may be implemented to ensure that each site within a multi-answer DNS response is apart of that domain."

\section{DNS Filtering}

Another technique used to prevent DNS rebinding attacks includes
filtering out private IP addresses from DNS responses, which can be
done in a number of ways.

\subsection{Filtering through external DNS servers}

OpenDNS is a service that extends the DNS by adding several security
features, in addition to regular DNS services. One of these features
includes optional content filtering and the ability to filter out
IP addresses from DNS responses. It supports a protocol which
authenticates traffic that moves between a host computer and the
OpenDNS nameservers, called the DNSCrypt protocol.

\subsection{Local nameserver configuration}

System admins can configure their organisation's local nameservers
such that external names cannot be resolved/mapped to the organisation's
internal IP addresses. This is a useful technique to prevent against
DNS rebinding, however a potential attacker could map the internal
IP address ranges that the organisation is currently using.

\subsection{DNS filtering in a firewall}

dnswall?

\section{Router Configuration}

DNS Rebinding attacks have been stopped more successfully through
more secure router configurations. Services, such as the HTTP server
on a router, are bound to all network interfaces and can be accessible
by all IP addresses in a range that it has. The routers drop anything
that comes in through the external port, which doesn't matter because
it accesses external IP addresses from the internal Local Area Network (LAN).

\section{Firefox NoScript Extension}

??
