\chapter{DNS Rebinding Countermeasures}

There have been attempts made to eradicate DNS rebinding, but only
a few methods have been proven effective at mitigating the effects
or stopping it entirely.

\section{Extended Same-Origin Policy}

With early iterations of the Same-Origin policy, many variations of
DNS rebinding attacks would exploit the policy. To counter this, web
browsers implemented countermeasures (as described in this section) to
protect resources. The Same-Origin policy relies on information
obtained from the DNS, which may not be under control of the owners
of a web server. Therefore a light-weight extension to the policy,
called the Extended Same-Origin policy, was made to also take into
account information provided by a web server to avoid exploitation by
DNS rebinding~\cite{johns2013eradicating}.

\vspace{0.5cm}

Johns, Lekies and Stock describes the new Extended Same-Origin policy
as differing~\cite{johns2013eradicating} from the original policy by using input provided by the
web server, which is "delivered through an HTTP reponse header" in the
form of { protocol, domain, port, server-origin }. Since DNS rebinding is
a protocol-level flaw, the Extended Same-Origin policy adds more to the
prevention of SOP by using the provided server origin information in the
HTTP response header. If the header is not sent, then the browser will fall
back to using the original policy.

\section{DNS Pinning}

DNS Pinning is a technique web browsers implement, which locks an IP
address to the value provided by the initial DNS response. It has
however has been deprecated due to it unintentionally blocking a
few legitimate uses of Dynamic DNS. An example is load balancing,
which is a service provided by DNS. Load balancing is vital for
major web servers and DNS pinning interferes with this.
As well as this DNS pinning does not protect against \textbf{sophisticated} DNS
rebinding attacks, for example attacks that use iframes to subvert DNS pinning.

\section{DNS Filtering}

Another technique used to prevent DNS rebinding attacks includes
filtering out private IP addresses from DNS responses, which can be
done in a number of ways.

\subsection{Filtering through external DNS servers}

OpenDNS is a service that extends the DNS by adding several security
features, in addition to regular DNS services. One of these features
includes optional content filtering and the ability to filter out
IP addresses from DNS responses. It supports a protocol which
authenticates traffic that moves between a host computer and the
OpenDNS nameservers, called the DNSCrypt protocol.

\subsection{Local nameserver configuration}

System admins can configure their organisation's local nameservers
such that external names cannot be resolved/mapped to the organisation's
internal IP addresses. This is a useful technique to prevent against
DNS rebinding, however a potential attacker could map the internal
IP address ranges that the organisation is currently using.

Jackson et Al. describe their production of a C program, called
dnswall~\cite{jackson2009protecting}, which is used to prevent external
host names from being resolved to internal IP addresses. This is an effective
countermeasure as, with the prevention of resolving host names,
DNS rebinding cannot be used to bypass firewalls. More specifically,
dnswall "modifies DNS responses that attempt to bind external [host] names
to internal IP addresses".

\section{Application-level filtering}

Another preventative procedure to counter DNS rebinding is application-level
filtering. This involves writing application code to filter IP addresses
that are considered "acceptable". A pull request~\cite{transmission_pr} on the
Transmission Bittorrent client repository describes, in their application code,
a whitelist which allows localhost and ip addresses but refuses to accept
external host names.
