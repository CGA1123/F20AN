\chapter{Same-Origin Policy (SOP)}

The same-origin policy is a security mechanism of modern browsers, which
controls the communication between scripts running in a browser. Scripts that
are contained within a web page are permitted to access data in another web
page, so long as they both have the same \emph{origin}. The origin is defined
as a combination of:
\begin{itemize}
	\item{URI scheme (http(s), ftp, file, etc.)}
	\item{Host name}
	\item{Port number (21, 80, 9091, etc.)}
\end{itemize}

Without this policy, an attacker can obtain access to sensitive data. For
example, assume that an individual is using a website that handles personal data
such as a social networking site. Without SOP, and assuming the individual
visits a malicious website in another browser tab, JavaScript on that website
can do anything on the social network account that the person would be able to
do, such as reading private messages and analysing the HTML DOM-tree after
the person entered their password before submitting the form.

Taking this example, and applying it to a scenario involving online banking, it
can be easily understood that the policy is essential.
