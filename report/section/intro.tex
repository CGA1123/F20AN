\chapter{Introduction}

For the coursework assignment, we decided to study DNS rebinding attacks.
A DNS rebinding attack is an exploit in which an attacker subverts
the same-origin policy of browsers, by running a client-side script used to
attack target machines on a network, and converts them into open network
proxies\textsuperscript{\cite{jackson2009protecting}}. This allows attackers to
breach private networks, as well as use a victim machine for distributed
denial-of-service (DDoS) attacks amongst other malicious activities.

\vspace{0.5cm}

This report describes our implementation of a DNS rebinding
attack\textsuperscript{\cite{cve20185702}} which is demonstrated using two
Virtual Machines (VMs) running Ubuntu 17.10, where one of the VMs plays the role
of an attacker and the other, the role of a victim.
