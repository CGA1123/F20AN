\chapter{Introduction}

For the coursework assignment, we decided to study DNS rebinding attacks.
A DNS rebinding attack is an exploit in which an attacker subverts
the same-origin policy of browsers, by running a client-side script used to
attack target machines on a network, and converts them into open network
proxies~\cite{jackson2009protecting}. This allows attackers to
breach private networks, as well as use a victim machine for distributed
denial-of-service (DDoS) attacks amongst other malicious activities.

\vspace{0.5cm}

We use a DNS rebinding attack to exploit a vulnerability in the Transmission
BitTorrent client (pre v2.9.3), which leads to the execution of
arbitrary code on the target machine running the Bittorrent client. This report
describes our implementation\footnote{Available at:
\url{https://github.com/CGA1123/F20AN}} of CVE-2018-5702~\cite{cve20185702}
a vulnerability which can be exploited through a DNS Rebinding attack.
