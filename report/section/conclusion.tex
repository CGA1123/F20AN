\chapter{Conclusion}

DNS Rebinding attacks are interesting mostly because many application
developers are not aware of them, and do not take into account that services
running on a private network can still be accessed by external network by this
method. Many large applications (such as Transmission~\cite{tavis_report}, the
Battle.net Update Agent~\cite{battle_net_vuln},
$\mu$Torrent~\cite{utorrent_vuln})
have been found to be vulnerable to being exploited through this method, even
though this type of vulnerability has been known for almost over a decade
now~\cite{jackson2009protecting}.

While our implementation focuses on the Transmission BitTorrent client, the
implementation is rather general and could be changed with relative ease to
send requests to a different service. What is also interesting is that this
type of attack could conceptually very easily be done on websites without
their explicit knowledge, either perhaps though advertisements or by exploiting
a cross-site scripting attack.

Great care needs to be taken by users and developers when using/creating
applications that offer a web client to manage a local application, this type
of design which allows for great customisation of interfaces through the use of
HTML, CSS, and JavaScript, may also lead to exploitation through the methods
described in this project.

Because this is a protocol level `flaw', it relies heavily on application
developers to become aware of this potential vulnerability and develop
safeguards against it. While the DNS filtering countermeasures discussed in
\S\ref{dns_filtering} may be suitable for some networks or organisations, there
may be legitimate reason why a hostname may need to resolve to private IP
addresses, meaning this cannot be a universal solution to this problem.
